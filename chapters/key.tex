\chapter{Key Supporting Arguments}

Here are some key supporting arguments regarding China's 
technology and economic growth

\section{Export-Led Growth}

China's economic growth has been significantly driven by its 
exports. The country has become known as the "world's factory," 
manufacturing a wide range of goods for global markets. This 
export-oriented strategy has boosted economic development 
and job creation.

\section{Investment in Infrastructure}

China has invested heavily in infrastructure development. 
Projects like high-speed rail networks, airports, and ports 
have improved connectivity, making it easier to transport 
goods and people, thereby enhancing economic efficiency.

China has accelerated infrastructure investment in the first 
quarter of this year to propel economic growth, launching 
more than 10,000 projects throughout the country.
Analysts estimated that infrastructure investment grew 10 
percent year-on-year in the first three months, driving up 
activity of many associated downstream enterprises and broad 
market demand for basic materials.
According to incomplete statistics, 14 provinces had announced 
data on major projects for the first quarter as of Monday, 
launching a total of 12,571 major projects in sectors including 
transport, water conservation, advanced manufacturing, modern 
services and new types of infrastructure.
The combined investment reached approximately 7 trillion yuan 
(\$1.03 trillion), according to media reports.


\section{Innovation and Research \& Development}

China has increased investments in research and development 
(R\&D), leading to innovations in areas like telecommunications, 
artificial intelligence, and renewable energy. This commitment 
to innovation has enabled China to compete on the global stage 
in technology and other high-value sectors.

China has leaned on its manufacturing prowess for decades to 
support economic development, but it is increasingly seeking 
to contend with countries whose economies are deeply rooted in 
innovation-based growth. China has made considerable progress 
in establishing itself as a pioneer in emerging industries and 
its leaders are increasingly looking toward innovation as a 
driver of its economic growth.

\section{Government Policies}

The Chinese government has played a significant role in fostering 
economic growth and technological advancement. Policies such as 
"Made in China 2025" and "Belt and Road Initiative" have been 
instrumental in guiding China's development and influence on the 
world stage.

In the city of Shanghai, a few churches conduct daily services 
for the faithful, just as churches all over the world do. However, 
China's Patriotic Catholic Association doesn't operate under the 
auspices of the Roman Catholic Church, which the Chinese government 
has banned. It is controlled by a state agency, the Religious 
Affairs Bureau. That's how the Chinese government deals with 
foreign organizations, be they churches or companies. They are 
tolerated in China but can operate only under the state's 
supervision. They can bring in their ideas if they deliver 
value to the country, but their operations will be circumscribed 
by China's goals. If the value—or danger—from them is high, the 
government will create hybrid organizations that it can better 
control. This approach, which never ceases to shock foreigners, 
guides those who are boldly fashioning a new China.

\section{Global Trade and Integration}

China's active participation in global trade, membership in 
international organizations like the World Trade Organization, 
and its position in global supply chains have enhanced its economic 
growth and global influence.

China's engagement in the so-called international fragmentation of 
production - namely 'cross-border dispersion of component 
production/assembly within vertically integrated manufacturing 
industries' - has become an increasingly important form of its 
economic integration into the regional as well as the global 
economy. The paper presents the recent trend of trade in parts 
and components between China and its main trading partners. 
Applying an adjusted gravity modelling method, the paper explores 
how China's pattern of trade in parts and components is being 
determined. The paper found that China's rapid economic growth, 
increasing market size and economies of scale, foreign direct 
investment and infrastructure development including transportation 
and telecommunications are important factors in explaining China's 
rapid increase of bilateral trade in parts and components with its 
trading partners. The paper also found that the spatial distance 
and transportation costs have significant negative impacts on China's 
trade of parts and components suggesting that the reduction in 
transportation costs by technological innovation and investment 
could enhance trade in parts and components, and thereby deepen 
the process of international specialization involving China and 
its main trading partners. The paper argues that given the 
prospects of the rapid growth of the Chinese economy, its current 
and planned massive investments in R\&D and in infrastructure, its 
continual policies in attracting FDI and its rapid move towards 
liberalizing its services sectors including its financial sectors, 
the scope for China and its trading partners to benefit from the 
process of international fragmentation of production is tremendous.

\section{Education and Workforce}

China has invested in education and skills development, resulting 
in a highly skilled and competitive workforce. This has attracted 
foreign companies and stimulated domestic innovation.

After decades of reform, China today has an education system that 
serves the industrial economy well although gaps in access, quality, 
and relevance in education still need to be plugged. However, there 
is now an even larger challenge to meet: delivering the skills needed 
for a modern, digital, postindustrial economy, while instilling a 
new national ethos of lifelong learning, and ensuring that the system 
is equitable. Nothing less than a transformation of China's education 
and skills-development system appears necessary. China has undertaken 
transformative reform before; it now needs to do so again.

\section{Urbanization}

China has experienced massive urbanization, leading to the growth 
of megacities and urban clusters. Urbanization has driven economic 
activity and increased consumer demand, contributing to 
overall growth.

By the same token, urbanization rarely exceeded ten percent of the 
total population although large urban centres were established. For 
example, during the Song, the northern capital Kaifeng 
(of the Northern Song) and southern capital Hangzhou 
(of the Southern Song) had 1.4 million and one million inhabitants, 
respectively. In addition, it was common that urban residents 
also had one foot in the rural sector due to private landholding 
property rights.

In 1949, the year that the People's Republic of China was founded, 
less than 10\% of the population in mainland China was urban. 
Few cities at that time could be considered modern.

\section{Technological Dominance}

China has gained global dominance in specific technological areas, 
such as 5G technology and electric vehicles, further propelling its 
economic growth and influence.

Building technological innovation is a gradual and cumulative process 
driven by industrial R\&D. China has a relatively short history of 
industrial innovation, which is path-dependent. For this reason, 
China has few advantages in established industries such as 
semiconductors and pharmaceuticals, where Western incumbents hold 
'patent thickets' that curb China's catch-up. While China contributed 
27.5 per cent to total global R\&D expenditures in 2022 against the 
United States' 35.6 per cent, US technology giants still dominate 
research and innovation in critical technologies such as artificial 
intelligence.