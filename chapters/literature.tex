\chapter{Literature Review}

Many studies show China`technology and economic growth of 
china are rapidly developing. Moreover, China`s rapid economic 
and technological growth can be attributed to a combination of 
various factors, policies, and strategic decisions. 

\section{The impact of China`s R\&D subsidies on R\&D investment, technological upgrading and economic growth}

A key argument in favor of state-intervention relates to the idea that 
investments in innovation are limited by financial constraints facing 
firms,3 especially ones from transitioning economy countries~\cite{BOEING2022121212}.

To better harness the growth-enhancing power of innovation, an 
important question that naturally arises for policy-makers is 
how deeply should the state intervene in promoting a country's 
own technological capabilities. Stemming from the failed import 
substitution policies of the 1970s, conventional wisdom calls for 
a rather limited role of the government to support indigenous 
innovation efforts given the public good nature of research and 
development. 2Yet, the explosion of innovation activities in 
emerging economies coinciding with periods of rapid economic 
growth has led to a renewed optimism that state-led innovation 
can be a major contribution to stimulate regional innovation 
systems and national competitive advantage.

\section{China`s changing political landscape: prospects for democracy}

Although each chapter makes equally important and meaningful 
contributions to this impressively coherent volume, readers 
will notice that agency, in the process of democratization, 
is the most salient issue.

Thirty years ago Deng Xiaoping launched his policy of 
“Reform and Opening.” In time, his decision would transform China 
economically, socially, legally, ideologically, and politically, 
no less than Mao's revolution did in 1949. The changes unleashed 
by Deng are difficult to overstate; they did nothing less than 
bring China for the first time fully into the modern world. The 
result is the nation of today's headlines: the third largest 
economy in the world; a land of 200 million Internet users and 
500 million cell phones; a significant actor in some of the most 
pressing international concerns (North Korea, Iran, Africa).

\section{Regional Income Inequality and Economic Growth in China}

Convergence is conditional on physical investment share, employment 
growth, human-capital investment, foreign direct investment, and 
coastal location. We project that, in the near term, overall regional 
inequality as measured by the coefficient of variation is likely to 
decline modestly but that the coast/noncoast income differential is 
likely to increase somewhat~\cite{chen1996regional}

China's spatial income inequality can be defined by inequality 
among regions and urban-rural income disparity. Certain regions, 
especially in eastern China, have more disproportionate advantages 
from the reform and opening up because of preferential policies, 
natural endowment, and improved infrastructure. Compared with the 
central and western areas, the income level in the east is higher, 
resulting in income inequality among regions. Inequality among 
regions and urban-rural income disparity are not entirely 
independent, suggesting that the difference in regional development 
also promotes the further differentiation of urban and rural 
development levels.